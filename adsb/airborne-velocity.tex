\section{Airborne Velocity}\label{airborne-velocity}

There are two different types of messages for velocities, determined by
3-bit subtype in the message. With subtype 1 and 2, surface velocity
(ground speed) is reported. And in subtype 3 and 4, aircraft airspeed is
reported.

Type 2 and 4 are for supersonic aircraft. So, before we have another
commercial supersonic aircraft flying around, you won't see any of those
types.

In real world, very few of subtype 3 messages are reported. In our
setup, we only received \textbf{0.3\%} of these message with regards to
subtype 1.

An aircraft velocity message has \texttt{DF:\ 17\ or\ 18},
\texttt{TC:\ 19}, and the subtype codes are represented in bits 38 to
40. Now, we can decode those messages.

\subsection{Subtype 1 (Ground Speed)}\label{subtype-1-ground-speed}

Subtype 1 (subsonic, ground speed), are broadcast when ground velocity
information is available. The aircraft velocity contains speed and
heading information. The speed and heading are also decomposed into
North-South, and East-West components.

For example, the following message is received:

\begin{verbatim}
Message: 8D485020994409940838175B284F

|    | ICAO24 |      DATA      |  CRC   |
|----|--------|----------------|--------|
| 8D | 485020 | 99440994083817 | 5B284F |

Convert DATA [99440994083817] into binary:

|-------|-----|----|--------|-----|------|------------|
|  TC   | ST  | IC | RESV_A | NAC | S-EW | V-EW       |
|-------|-----|----|--------|-----|------|------------|
| 10011 | 001 | 0  | 1      | 000 | 1    | 0000001001 |

|------|------------|-------|------|-----------|--------|-------|---------|
| S-NS | V-NS       | VrSrc | S-Vr | Vr        | RESV_B | S_Dif | Dif     |
|------|------------|-------|------|-----------|--------|-------|---------|
| 1    | 0010100000 | 0     | 1    | 000001110 | 00     | 0     | 0010111 |
\end{verbatim}

There are quite a few parameters in the velocity message. From left to
right, the number of bits indicates the contents in following Table \ref{tb:adsb-v-bits-st-1}

\begin{table}[!ht]
\centering
\caption{Airborne velocity message bits explained - Subtype 1}
\label{tb:adsb-v-bits-st-1}
\begin{tabular}{@{}lllll@{}}
\toprule
MSG Bits & Data Bits & Len & Abbr    & Content                    \\ \midrule
33 - 37  & 1 - 5     & 5   & TC      & Type code                  \\
38 - 40  & 6 - 8     & 3   & ST      & Subtype                    \\
41       & 9         & 1   & IC      & Intent change flag         \\
42       & 10        & 1   & RESV\_A & Reserved-A                 \\
43 - 45  & 11 - 13   & 3   & NAC     & Velocity uncertainty (NAC) \\
46       & 14        & 1   & S\_ew   & East-West velocity sign    \\
47 - 56  & 15 - 24   & 10  & V\_ew   & East-West velocity         \\
57       & 25        & 1   & S\_ns   & North-South velocity sign  \\
58 - 67  & 26 - 35   & 10  & V\_ns   & North-South velocity       \\
68       & 36        & 1   & VrSrc   & Vertical rate source       \\
69       & 37        & 1   & S\_vr   & Vertical rate sign         \\
70 - 78  & 38 - 46   & 9   & Vr      & Vertical rate              \\
79 - 80  & 47 - 48   & 2   & RESV\_B & Reserved-B                 \\
81       & 49        & 1   & S\_Dif  & Diff from baro alt, sign   \\
82 - 88  & 50 - 56   & 7   & Dif     & Diff from baro alt         \\ \bottomrule
\end{tabular}
\end{table}


\subsubsection{Horizontal Velocity}\label{horizontal-velocity}

For calculating the horizontal speed and heading we need four values,
East-West Velocity \texttt{V\_ew}, East-West Velocity Sign
\texttt{S\_ew}, North-South Velocity \texttt{V\_ns}, North-South
Velocity Sign \texttt{S\_ns}. And pay attention on the directions
(signs) in the calculation.

\begin{verbatim}
S_ns:
    1 -> flying North to South
    0 -> flying South to North
S_ew:
    1 -> flying East to West
    0 -> flying West to East
\end{verbatim}

The Speed (v) and heading (h) with unit knot and degree can be computed
as follows:

\[V_{we} =
\begin{cases}
 -1 \cdot (V_{ew} - 1)    & \text{if } s_{ew} = 1 \\
 V_{ew} - 1         & \text{if } s_{ew} = 0
\end{cases}\]

\[V_{sn} =
\begin{cases}
 -1 \cdot (V_{ns} - 1)    & \text{if } s_{ns} = 1 \\
 V_{ns} - 1         & \text{if } s_{ns} = 0
\end{cases}\]

\[v = \sqrt{V_{we}^{2} + V_{sn}^{2}}\]

\[h = arctan2 \left( V_{we}, V_{sn} \right) \cdot \frac{360}{2\pi}  \quad \text{(deg)}\]

In case of a negative value here, we will simply add 360 degrees.

\[h = h + 360  \quad (\text{if } h < 0)\]

So, now we have the speed and heading of our example:

\begin{verbatim}
V-EW: 0000001001 -> 9
S-EW: 1
V-NS: 0010100000 -> 160
S-NS: 1

V_{we} = - (9 - 1) = -8
V_{sn} = - (160 - 1) = -159

v = 159.20 (kt)
h = 182.88 (deg)
\end{verbatim}

\subsubsection{Vertical Rate}\label{vertical-rate}

The direction of vertical movement of aircraft can be read from
\texttt{S\_vr} field, in message bit-69:

\begin{verbatim}
0 -> UP
1 -> Down
\end{verbatim}

The actual vertical rate \texttt{Vr} is the value of bits 70-78, minus
1, and then multiplied by 64 in \textbf{feet/minute} (ft/min). In our
example:

\begin{verbatim}
Vr-bits: 000001110 = 14
Vr: (14 - 1) x 64 => 832 fpm
S-Vr: 0 => Down / Descending
\end{verbatim}

So we see a descending aircraft at 832 ft/min rate of descend.

The Vertical Rate Source (VrSrc) field determines if it is a measurement
in barometric pressure altitude or geometric altitude:

\begin{verbatim}
0 ->  Baro-pressure altitude change rate
1 ->  Geometric altitude change rate
\end{verbatim}

\subsection{Subtype 3 (Airspeed)}\label{subtype-3-airspeed}

Subtype 3 (subsonic, airspeed), are broadcast when ground speed
information is NOT available, while airspeed is available. The structure
of the message is similar to the previous one. Let's take a close look
at an example for decoding here.

\begin{verbatim}
Message: 8DA05F219B06B6AF189400CBC33F

|    | ICAO24 |      DATA      |  CRC   |
|----|--------|----------------|--------|
| 8D | A05F21 | 9B06B6AF189400 | CBC33F |

Convert DATA [9B06B6AF189400] into binary:

|-------|-----|----|--------|-----|------|------------|------|------------|
|  TC   | ST  | IC | RESV_A | NAC | S_hdg| Hdg        | AS-t | AS         |
|-------|-----|----|--------|-----|------|------------|------|------------|
| 10011 | 011 | 0  | 0      | 000 | 1    | 1010110110 | 1    | 0101111000 |

|-------|------|-----------|--------|-------|---------|
| VrSrc | S-Vr | Vr        | RESV_B | S_Dif | Dif     |
|-------|------|-----------|--------|-------|---------|
| 1     | 1    | 000100101 | 00     | 0     | 0000000 |
\end{verbatim}

The details on all data bits are described in following Table \ref{tb:adsb-v-bits-st-3}:

\begin{table}[]
\centering
\caption{Airborne velocity message bits explained - Subtype 3}
\label{tb:adsb-v-bits-st-3}
\begin{tabular}{@{}lllll@{}}
\toprule
MSG Bits   & DATA Bits   & Len & Abbr    & Content                        \\ \midrule
33 - 37    & 1 - 5       & 5   & TC      & Type code                      \\
38 - 40    & 6 - 8       & 3   & ST      & Subtype                        \\
41         & 9           & 1   & IC      & Intent change flag             \\
42         & 10          & 1   & RESV\_A & Reserved-A                     \\
43 - 45    & 11 - 13     & 3   & NAC     & Velocity uncertainty (NAC)     \\
46         & 14          & 1   & S\_hdg  & Heading status                 \\
47 - 56    & 15 - 24     & 10  & Hdg     & Heading (proportion)           \\
57         & 25          & 1   & AS-t    & Airspeed Type                  \\
58 - 67    & 26 - 35     & 10  & AS      & Airspeed                       \\
68         & 36          & 1   & VrSrc   & Vertical rate source           \\
69         & 37          & 1   & S\_vr   & Vertical rate sign             \\
70 - 78    & 38 - 46     & 9   & Vr      & Vertical rate                  \\
79 - 80    & 47 - 48     & 2   & RESV\_B & Reserved-B                     \\
81         & 49          & 1   & S\_Dif  & Difference from baro alt, sign \\
82 - 88    & 50 - 66     & 7   & Dif     & Difference from baro alt       \\ \bottomrule
\end{tabular}
\end{table}

\subsubsection{Heading}\label{heading}

\texttt{S\_hdg} makes the status of heading data:

\begin{verbatim}
0 -> heading data not available
1 -> heading data available
\end{verbatim}

10-bits \texttt{Hdg} is the represent the proportion of the degrees of a
full circle, i.e. 360 degrees. (Note: 0000000000 - 1111111111 represents
0 - 1023 )

\[heading = Decimal(Hdg) / 1024 * 360^o\]

in our example :

\begin{verbatim}
1010110110 -> 694
heading = 694 / 1024 * 360 = 243.98 (degree)
\end{verbatim}

\subsubsection{Velocity (Airspeed)}\label{velocity-airspeed}

To find out which type of the airspeed (TAS or IAS), first we need to
look at the \texttt{AS-t} field:

\begin{verbatim}
0 -> Indicated Airspeed (IAS)
1 -> True Airspeed (TAS)
\end{verbatim}

And then the speed is simply a binary to decimal conversion of
\texttt{AS} bits (in knot). In our example:

\begin{verbatim}
0101111000 -> 376 knot
\end{verbatim}

\subsubsection{Vertical Rate}\label{vertical-rate-1}

The vertical rate decoding remains the same as subtype 1.
