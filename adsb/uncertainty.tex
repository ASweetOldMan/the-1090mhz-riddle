\section{Uncertainty and accuracy}\label{uncertainty-and-accuracy}

NIC, NAC, NUC, and SIL, those acronyms do sound confusing. They are
categorical numbers for the integrity, accuracy, or uncertainties of the
position measurements.

\begin{itemize}

\item
  \texttt{NUCp}: Navigation Uncertainty Category - Position

  \begin{itemize}

  \item
    Values: 0 - 9
  \item
    Version 0, 1, and 2
  \end{itemize}
\item
  \texttt{NUCr}: Navigation Uncertainty Category - Velocity (Rate)

  \begin{itemize}

  \item
    Values: 0 - 4
  \item
    Version 0
  \end{itemize}
\item
  \texttt{NIC}: Navigation Integrity Category

  \begin{itemize}

  \item
    Values: 0 - 11
  \item
    Version 1 and 2
  \end{itemize}
\item
  \texttt{NACp}: Navigation Accuracy Category - Position

  \begin{itemize}

  \item
    Values: 0 - 11
  \item
    Version 1 and 2
  \end{itemize}
\item
  \texttt{NACv}: Navigation Accuracy Category - Velocity

  \begin{itemize}

  \item
    Values: 0 - 4
  \item
    Version 1 and 2
  \end{itemize}
\item
  \texttt{SIL}: Surveillance Integrity Level

  \begin{itemize}

  \item
    Values: 0 - 3
  \item
    Version 1 and 2
  \end{itemize}
\end{itemize}

For each category name, specific values are given corresponding to the
numerical indicators. The relation of the category names and value names
are:

\begin{itemize}

\item
  \texttt{NUCp}:

  \begin{itemize}

  \item
    Horizontal Protection Limit (\texttt{HPL})
  \item
    95\% Containment Radius - Horizontal (\texttt{RCu})
  \item
    95\% Containment Radius - Vertical (\texttt{RCv})
  \end{itemize}
\item
  \texttt{NUCr}:

  \begin{itemize}

  \item
    95\% Horizontal Velocity Error (\texttt{HVE})
  \item
    95\% Vertical Velocity Error (\texttt{VVE})
  \end{itemize}
\item
  \texttt{NIC}:

  \begin{itemize}

  \item
    Horizontal Radius of Containment (\texttt{RCu})
  \item
    Vertical Protection Limit (\texttt{VPL})

    \begin{itemize}

    \item
      a.k.a. Integrity Containment Region
    \end{itemize}
  \end{itemize}
\item
  \texttt{NACp}:

  \begin{itemize}

  \item
    95\% horizontal accuracy bounds, Estimated Position Uncertainty
    (\texttt{EPU})

    \begin{itemize}

    \item
      a.k.a. Horizontal Figure of Merit (\texttt{HFOM})
    \end{itemize}
  \item
    95\% vertical accuracy bounds, Vertical Estimated Position
    Uncertainty (\texttt{VEPU})

    \begin{itemize}

    \item
      a.k.a. Vertical Figure of Merit (\texttt{VFOM})
    \end{itemize}
  \end{itemize}
\item
  \texttt{NACv}:

  \begin{itemize}

  \item
    95\% horizontal accuracy bounds for velocity, Horizontal Figure of
    Merit (\texttt{HFOMr})
  \item
    95\% vertical accuracy bounds for velocity, Vertical Figure of Merit
    (\texttt{VFOMr})
  \end{itemize}
\item
  \texttt{SIL}:

  \begin{itemize}

  \item
    Probability of exceeding Horizontal Radius of Containment
    \texttt{RCu} (\texttt{PE\_RCu})
  \item
    Probability of exceeding Vertical Integrity Containment Region
    \texttt{VPL} (\texttt{PE\_VPL})
  \end{itemize}
\end{itemize}

Depending on the ADS-B versions, the bits to uncover these values maybe
different. We are going to address the uncertainty measures by ADS-B
versions.

\begin{quote}
To understand about these versions, first take a look at the
\href{version.html}{ADS-B version chapter}.
\end{quote}

\subsection{Version 0}\label{version-0}

\subsubsection{NUCp}\label{nucp}

In ADS-B \texttt{Version\ 0}, the accuracy is expressed as
\textbf{Navigation Uncertainty Category - position} (\texttt{NUCp}). It
is directly related (one-to-one relationship) with \texttt{Type\ Code},
as follows:

For surface position messages:

\begin{verbatim}
+------+---+----+----+----+---+
| TC   | 0 | 5  | 6  | 7  | 8 |
+------+---+----+----+----+---+
| NUCp | 0 | 9  | 8  | 7  | 6 |
+------+---+----+----+----+---+
\end{verbatim}

For airborne position with barometric altitude:

\begin{verbatim}
+------+---+----+----+----+----+----+----+----+----+----+
| TC   | 9 | 10 | 11 | 12 | 13 | 14 | 15 | 16 | 17 | 18 |
+------+---+----+----+----+----+----+----+----+----+----+
| NUCp | 9 | 8  | 7  | 6  | 5  | 4  | 3  | 2  | 1  | 0  |
+------+---+----+----+----+----+----+----+----+----+----+
\end{verbatim}

For airborne position with GNSS altitude:

\begin{verbatim}
+------+------+------+-----+
| TC   |  20  |  21  | 22  |
+------+------+------+-----+
| NUCp |  9   |  8   | 0   |
+------+------+------+-----+
\end{verbatim}

Higher number of NUCp represents a higher confidence of in the position
measurement (hence, a lower uncertainty). Horizontal Protection Limit
(\texttt{HPL}) and Radius of containment for horizontal (\texttt{RCu})
are used to quantify the uncertainty.

For surface position (\texttt{TC=5-8}):

\begin{verbatim}
+----+------+--------------------+---------------------+
| TC | NUCp | HPL                | RCu                 |
+====+======+====================+=====================+
| 5  |  9   | < 7.5 m            | < 3 m               |
+----+------+--------------------+---------------------+
| 6  |  8   | < 25 m             | < 10 m              |
+----+------+--------------------+---------------------+
| 7  |  7   | < 0.1 NM (185 m)   | < 0.05 NM (93 m)    |
+----+------+--------------------+---------------------+
| 8  |  6   | > 0.1 NM (185 m)   | > 0.05 NM (93 m)    |
+----+------+--------------------+---------------------+
\end{verbatim}

For airborne position with barometric altitude (\texttt{TC=9-18}):

\begin{verbatim}
+------+------+--------------------+---------------------+
|  TC  | NUCp | HPL                | RCu                 |
+======+======+====================+=====================+
|  9   |  9   | < 7.5 m            | < 3 m               |
+------+------+--------------------+---------------------+
|  10  |  8   | < 25 m             | < 10 m              |
+------+------+--------------------+---------------------+
|  11  |  7   | < 0.1 NM (185 m)   | < 0.05 NM (93 m)    |
+------+------+--------------------+---------------------+
|  12  |  6   | < 0.2 NM (370 m)   | < 0.1 NM (185 m)    |
+------+------+--------------------+---------------------+
|  13  |  5   | < 0.5 NM (926 m)   | < 0.25 NM (463 m)   |
+------+------+--------------------+---------------------+
|  14  |  4   | < 1 NM (1852 m)    | < 0.5 NM (926 m)    |
+------+------+--------------------+---------------------+
|  15  |  3   | < 2 NM (3704 m)    | < 1 NM (1852 m)     |
+------+------+--------------------+---------------------+
|  16  |  2   | < 10 NM (18520 m)  | < 5 NM (9260 m)     |
+------+------+--------------------+---------------------+
|  17  |  1   | < 20 NM (37040 m)  | < 10 NM (18520 m)   |
+------+------+--------------------+---------------------+
|  18  |  0   | > 20 NM (37040 m)  | > 10 NM (18520 m)   |
+------+------+--------------------+---------------------+
\end{verbatim}

In the case of airborne position with GNSS height (\texttt{TC=20-22}),
\texttt{HPL} and \texttt{RCu} are defined slight differently. In
addition, Radius of containment for vertical position (\texttt{RCv}) is
added:

\begin{verbatim}
+------+------+----------+---------+---------+
|  TC  | NUCp | HPL      | RCu     | RCv     |
+======+======+==========+=========+=========+
|  20  |  9   | < 7.5 m  | < 3 m   | < 4 m   |
+------+------+----------+---------+---------+
|  21  |  8   | < 25 m   | < 10 m  | < 15 m  |
+------+------+----------+---------+---------+
|  22  |  0   | > 25 m   | > 10 m  | > 15 m  |
+------+------+----------+---------+---------+
\end{verbatim}

\subsubsection{NUCv}\label{nucv}

The \textbf{Navigation Uncertainty Category - velocity} (\texttt{NUCv})
it is used to indicate the uncertainty of the horizontal and vertical
speeds. Related bits are located at the airborne velocity message,
\texttt{TC=19}, message bit 43-45 (or payload bit 11-13). It defines the
95\% of the error in horizontal and vertical speed.

\begin{verbatim}
+------+-------------+------------------------+
| NUCp | HVE (95%)   | VVE (95%)              |
+======+=============+========================+
|  0   | unknown     | unknown                |
+------+-------------+------------------------+
|  1   | < 10 m/s    | < 15.2 m/s (50 pfs)    |
+------+-------------+------------------------+
|  2   | < 3 m/s     | < 4.5 m/s (15 fps)     |
+------+-------------+------------------------+
|  3   | < 1 m/s     | < 1.5 m/s (5 fps)      |
+------+-------------+------------------------+
|  4   | < 0.3 m/s   | < 0.46 m/s (1.5 fps)   |
+------+-------------+------------------------+
\end{verbatim}

\subsection{Version 1}\label{version-1}

\subsubsection{NIC}\label{nic}

In ADS-B Version 1, \textbf{Navigation Integrity Category}
(\texttt{NIC}) is introduced to provide more levels of uncertainty
definitions. The \texttt{NUCp} value is still kept, but has been moved
to the new ``operation status messages'' (\texttt{TC=31}).

As for \texttt{NIC}, in \texttt{Type\ Code}: \texttt{7} and \texttt{11},
two \texttt{NIC} levels present in each code. In order to distinguish
these two different levels, a \textbf{NIC Supplement Bit}
(\texttt{NICs}) is introduced in ``operation status messages''
\texttt{TC=31} (message bit 76 or payload bit 44). The relation of TC,
NIC, and Rc are list in following tables.

For surface position (\texttt{TC=5-8})

\begin{verbatim}
+------+--------+-------+-----------------------+
|  TC  | NICs   |  NIC  | Rc                    |
+======+========+=======+=======================+
|  5   | 0      |  11   | < 7.5 m               |
+------+--------+-------+-----------------------+
|  6   | 0      |  10   | < 25 m                |
+------+--------+-------+-----------------------+
|  7   | 1      |  9    | < 75 m                |
|      +--------+-------+-----------------------+
|      | 0      |  8    | < 0.1 NM (185 m)      |
+------+--------+-------+-----------------------+
|  8   | 0      |  0    | > 0.1 NM or Unknown   |
+------+--------+-------+-----------------------+
\end{verbatim}

For airborne position with barometric altitude (\texttt{TC=9-18}):

\begin{verbatim}
+----+--------+-----+--------------------+----------+
| TC | NICs   | NIC | Rc                 | VPL      |
+====+========+=====+====================+==========+
| 9  | 0      | 11  | < 7.5 m            | < 11 m   |
+----+--------+-----+--------------------+----------+
| 10 | 0      | 10  | < 25 m             | < 37.5 m |
+----+--------+-----+--------------------+----------+
| 11 | 1      | 9   | < 75 m             | < 112 m  |
|    +--------+-----+--------------------+----------+
|    | 0      | 8   | < 0.1 NM (185 m)   |          |
+----+--------+-----+--------------------+----------+
| 12 | 0      | 7   | < 0.2 NM (370 m)   |          |
+----+--------+-----+--------------------+----------+
| 13 | 0      | 6   | < 0.5 NM (926 m)   |          |
|    +--------+     +--------------------+----------+
|    | 1      |     | < 0.6 NM (1111 m)  |          |
+----+--------+-----+--------------------+----------+
| 14 | 0      | 5   | < 1.0 NM (1852 m)  |          |
+----+--------+-----+--------------------+----------+
| 15 | 0      | 4   | < 2 NM (3704 m)    |          |
+----+--------+-----+--------------------+----------+
| 16 | 1      | 3   | < 4 NM (7408 m)    |          |
|    +--------+-----+--------------------+----------+
|    | 0      | 2   | < 8 NM (14.8 km)   |          |
+----+--------+-----+--------------------+----------+
| 17 | 0      | 1   | < 20 NM (37.0 km)  |          |
+----+--------+-----+--------------------+----------+
| 18 | 0      | 0   | > 20 NM or Unknown |          |
+----+--------+-----+--------------------+----------+
\end{verbatim}


In the case of airborne position with GNSS height (\texttt{TC=20-22}),
\texttt{Rc} is defined slight differently:

\begin{verbatim}
+------+------+----------+----------+
|  TC  | NIC  | Rc       | VPL      |
+======+======+==========+==========+
|  20  | 11   | < 7.5 m  | < 11 m   |
+------+------+----------+----------+
|  21  | 10   | < 25 m   | < 37.5 m |
+------+------+----------+----------+
|  22  | 0    | > 25 m   | > 112 m  |
+------+------+----------+----------+
\end{verbatim}

\subsubsection{NACp}\label{nacp}

In addition to \texttt{NIC}, \textbf{Navigation Accuracy Category -
Position} (\texttt{NACp}) is also introduced int ADS-B
\texttt{Version\ 1}. \texttt{NACp} can be found in:

\begin{itemize}

\item
  \texttt{TC=29}, target state and status message, message bit 72-75 (or
  payload bit 40-43)
\item
  \texttt{TC=31}, operational status message, message bit 77-80 (or
  payload bit 45-48)
\end{itemize}

With \texttt{NACp}, one can find out the 95\% horizontal and vertical
accuracy bounds (\texttt{EPU} and \texttt{VEPU}, or \texttt{HFOM} and
\texttt{VFOM}). \texttt{NACp} and \texttt{NIC} usually share the same
value, and their values are closely related.

\begin{equation}
  \begin{split}
    \mathrm{EPU} &\approx \mathrm{Rc} / 2.5   \qquad  \mathrm{NAC, NIC} \ge 9 \\
    \mathrm{EPU} &\approx \mathrm{Rc} / 2.0  \qquad  \mathrm{NAC, NIC} > 9
  \end{split}
\end{equation}

For ADS-B, \texttt{NACp} is also indicated in the operational status
messages (\texttt{TC=31}). \texttt{NACp} and corresponding
\texttt{EPU}/\texttt{VEPU} are:

\begin{verbatim}
+------+--------------------+-------------+
| NACp | EPU (HFOM)         | VEPU (VFOM) |
+======+====================+=============+
|  11  | < 3 m              | < 4 m       |
+------+--------------------+-------------+
|  10  | < 10 m             | < 15 m      |
+------+--------------------+-------------+
|  9   | < 30 m             | < 45 m      |
+------+--------------------+-------------+
|  8   | < 0.05 NM (93 m)   |             |
+------+--------------------+-------------+
|  7   | < 0.1 NM (185 m)   |             |
+------+--------------------+-------------+
|  6   | < 0.3 NM (556 m)   |             |
+------+--------------------+-------------+
|  5   | < 0.5 NM (926 m)   |             |
+------+--------------------+-------------+
|  4   | < 1.0 NM (1852 m)  |             |
+------+--------------------+-------------+
|  3   | < 2 NM (3704 m)    |             |
+------+--------------------+-------------+
|  2   | < 4 NM (7408 m)    |             |
+------+--------------------+-------------+
|  1   | < 10 NM (18520 km) |             |
+------+--------------------+-------------+
|  0   | > 10 NM or Unknown |             |
+------+--------------------+-------------+
\end{verbatim}


\subsubsection{NACv}\label{nacv}

The \textbf{Navigation Accuracy Category - Velocity} (\texttt{NACv}) is
introduced in \texttt{Version\ 1} to replace the \texttt{NUCv} from
\texttt{Version\ 0}. The bits are located at the same location and have
the same definitions of values. It can be found in airborne velocity
message, \texttt{TC=19}, message bit 43-45 (or payload bit 11-13). It
defines the 95\% of the error in horizontal and vertical speed.

\begin{verbatim}
+------+-------------+------------------------+
| NAVc | HFOMr (95%) | VFOMr (95%)            |
+======+=============+========================+
|  0   | unknown     | unknown                |
+------+-------------+------------------------+
|  1   | < 10 m/s    | < 15.2 m/s (50 pfs)    |
+------+-------------+------------------------+
|  2   | < 3 m/s     | < 4.5 m/s (15 fps)     |
+------+-------------+------------------------+
|  3   | < 1 m/s     | < 1.5 m/s (5 fps)      |
+------+-------------+------------------------+
|  4   | < 0.3 m/s   | < 0.46 m/s (1.5 fps)   |
+------+-------------+------------------------+
\end{verbatim}

\subsubsection{SIL}\label{sil}

On the other hand, \textbf{Surveillance Integrity Level} (\texttt{SIL})
is introduced in \texttt{Version\ 1} to indicate the probability of
measurement exceeding the containment radius horizontally
(\texttt{PE\_RCu}) and vertically (\texttt{PE\_VPL}).

\texttt{SIL} value can be found in two different locations:

\begin{itemize}

\item
  \texttt{TC=29}, target state and status message, message bit 77-78 (or
  payload bit 45-46)
\item
  \texttt{TC=31}, operational status message, message bit 83-84 (or
  payload bit 51-52)
\end{itemize}

The definitions are as follows:

\begin{verbatim}
+-----+--------------------+--------------------+
| SIL | PE_RCu             | PE_VPL             |
+=====+====================+====================+
| 0   | unknown            | unknown            |
+-----+--------------------+--------------------+
| 1   | < 1 x 10^-3        | < 1 x 10^-3        |
+-----+--------------------+--------------------+
| 2   | < 1 x 10^-5        | < 1 x 10^-5        |
+-----+--------------------+--------------------+
| 3   | < 1 x 10^-7        | < 2 x 10^-7        |
+-----+--------------------+--------------------+
\end{verbatim}

The unit for \texttt{PE\_RCu} and \texttt{PE\_VPL} is per hour or per
sample.

\subsection{Version 2}\label{version-2}

\subsubsection{NIC}\label{nic-1}

In ADS-B \texttt{Version\ 2}, levels of accuracies are further divided
by two additional \texttt{NIC} supplement bits. In \texttt{Version\ 2}, we
call them NIC Supplement Bit A (\texttt{NICa}), NIC Supplement Bit B
(\texttt{NICb}), and NIC Supplement Bit C (\texttt{NICc}).

\begin{itemize}

\item
  \texttt{NICa} is in the same location as in ADS-B \texttt{Version\ 1},
  which is located in the operational status message \texttt{TC=31}
  (message bit 76 or payload bit 44).
\item
  \texttt{NICb} is located in the airborne position message
  (\texttt{TC=9-18}, message bit 40 or payload bit 8), where the
  ``single antenna flag'' was located in previous ADS-B versions.
\item
  \texttt{NICc} is also located in the operational status message
  \texttt{TC=31} (message bit 52 or payload bit 20).
\end{itemize}

Gather the NIC supplement bits, together with the \texttt{Type\ Code},
accuracies are defined.

For surface possible message (\texttt{TC=5-8}), using \texttt{TC},
\texttt{NICa}, and \texttt{NICc}:

\begin{verbatim}
+------+------+------+-------+-----------------------+
|  TC  | NICa | NICc |  NIC  | Rc                    |
+======+======+======+=======+=======================+
|  5   | 0    | 0    |  11   | < 7.5 m               |
+------+------+------+-------+-----------------------+
|  6   | 0    | 0    |  10   | < 25 m                |
+------+------+------+-------+-----------------------+
|  7   | 1    | 0    |  9    | < 75 m                |
|      +------+------+-------+-----------------------+
|      | 0    | 0    |  8    | < 0.1 NM (185 m)      |
+------+------+------+-------+-----------------------+
|  8   | 1    | 1    |  7    | < 0.2 NM (370 m)      |
|      +------+------+-------+-----------------------+
|      | 1    | 0    |  6    | < 0.3 NM (556 m)      |
|      +------+------+       +-----------------------+
|      | 0    | 1    |       | < 0.6 NM (1111 m)     |
|      +------+------+-------+-----------------------+
|      | 0    | 0    |  0    | > 0.6 NM or Unknown   |
+------+------+------+-------+-----------------------+
\end{verbatim}

For airborne possible message (\texttt{TC=9-18}), using \texttt{TC},
\texttt{NICa}, and \texttt{NICc}:

\begin{verbatim}
+----+------+------+-----+-----------------------+
| TC | NICa | NICb | NIC | Rc                    |
+====+======+======+=====+=======================+
| 9  | 0    | 0    | 11  | < 7.5 m               |
+----+------+------+-----+-----------------------+
| 10 | 0    | 0    | 10  | < 25 m                |
+----+------+------+-----+-----------------------+
| 11 | 1    | 1    | 9   | < 75 m                |
|    +------+------+-----+-----------------------+
|    | 0    | 0    | 8   | < 0.1 NM (185 m)      |
+----+------+------+-----+-----------------------+
| 12 | 0    | 0    | 7   | < 0.2 NM (370 m)      |
+----+------+------+-----+-----------------------+
| 13 | 0    | 1    | 6   | < 0.3 NM (556 m)      |
|    +------+------+     +-----------------------+
|    | 0    | 0    |     | < 0.5 NM (926 m)      |
|    +------+------+     +-----------------------+
|    | 1    | 1    |     | < 0.6 NM (1111 m)     |
+----+------+------+-----+-----------------------+
| 14 | 0    | 0    | 5   | < 1.0 NM (1852 m)     |
+----+------+------+-----+-----------------------+
| 15 | 0    | 0    | 4   | < 2 NM (3704 m)       |
+----+------+------+-----+-----------------------+
| 16 | 1    | 1    | 3   | < 4 NM (7408 m)       |
|    +------+------+-----+-----------------------+
|    | 0    | 0    | 2   | < 8 NM (14.8 km)      |
+----+------+------+-----+-----------------------+
| 17 | 0    | 0    | 1   | < 20 NM (37.0 km)     |
+----+------+------+-----+-----------------------+
| 18 | 0    | 0    | 0   | > 20 NM or Unknown    |
+----+------+------+-----+-----------------------+
\end{verbatim}

In the case of airborne position with GNSS height (\texttt{TC=20-22}),
the table remains the same as previous version.

\begin{verbatim}
+------+------+----------+
|  TC  | NIC  | Rc       |
+======+======+==========+
|  20  | 11   | < 7.5 m  |
+------+------+----------+
|  21  | 10   | < 25 m   |
+------+------+----------+
|  22  | 0    | > 25 m   |
+------+------+----------+
\end{verbatim}

\subsubsection{NACp}\label{nacp-1}

NACp in \texttt{Version\ 2} remains the same as in \texttt{Version\ 1}.

\subsubsection{NACv}\label{nacv-1}

NACv in \texttt{Version\ 2} remains the same as in \texttt{Version\ 1}.

\subsubsection{SIL}\label{sil-1}

In \texttt{Version\ 2}, an additional \textbf{SIL supplement bit}
(\texttt{SILs}) is introduced to distinguish if the value is based on
per flight hour or per sample.

The same as in \texttt{Version\ 1}, \texttt{SIL} value can be found in
two different locations:

\begin{itemize}

\item
  \texttt{TC=29}, target state and status message, message bit 77-78 (or
  payload bit 45-46)
\item
  \texttt{TC=31}, operational status message, message bit 83-84 (or
  payload bit 51-52)
\end{itemize}

\texttt{SILs} bit can also be found at two different locations:

\begin{itemize}

\item
  \texttt{TC=29}, target state and status message, message bit 40 (or
  payload bit 8)
\item
  \texttt{TC=31}, operational status message, message bit 87 (or payload
  bit 55)
\item
  \texttt{SILs=0}: probability is ``per hour'' based
\item
  \texttt{SILs=1}: probability is ``per sample'' based
\end{itemize}

The same values from \texttt{Version\ 1} are kept:

\begin{verbatim}
+-----+--------------------+--------------------+
| SIL | PE_RCu             | PE_VPL             |
+=====+====================+====================+
| 0   | unknown            | unknown            |
+-----+--------------------+--------------------+
| 1   | < 1 x 10^-3        | < 1 x 10^-3        |
+-----+--------------------+--------------------+
| 2   | < 1 x 10^-5        | < 1 x 10^-5        |
+-----+--------------------+--------------------+
| 3   | < 1 x 10^-7        | < 2 x 10^-7        |
+-----+--------------------+--------------------+
\end{verbatim}
