\section{ADS-B versions}\label{ads-b-versions}

In this advanced chapter, we are going to look into different versions and evolution of the ADS-B.

Since the beginning of ADS-B, there have been three different versions (to my knowledge) implemented. The major reason for these updates is to enable more information (types of data) in ADS-B. Documentations on these versions and differences are quite far from user friendly. They are always presented in a very scattered fashion. Even the official \texttt{ICAO\_9871} document is confusing to read. I am going to try my best to put the pieces together in this chapter.

There are three versions implemented so far, starting from Version 0, then Version 1 around 2008 and Version 2 around 2012. Major changes in Version 1 and Version 2 are listed as follows:

From \texttt{Version\ 0} to \texttt{Version\ 1}:

\begin{itemize}
\item
  Added Type Code 28, 29, and 31 messages

  \begin{itemize}
    \item
    \texttt{TC=28}: Aircraft status - Emergency/priority status and ACAS
    RA Broadcast
  \item
    \texttt{TC=29}: Target state and status
  \item
    \texttt{TC=31}: Operational status
  \end{itemize}
\item
  Introduced the ``Navigation integrity category (\texttt{NIC})'' and   ``Surveillance integrity level (\texttt{SIL})'' in addition to the   ``Navigation accuracy category (\texttt{NAC})'' from the \texttt{Version\ 0}

  \begin{itemize}
    \item
    Type Code and an NIC Supplement bit (\texttt{NICs}) is used to
    define the NIC
  \item
    NIC Supplement bit included in \texttt{TC=31} messages
  \end{itemize}
\item
  The ADS-B version number is now indicated in operation status message
  \texttt{TC=31}
\end{itemize}

From \texttt{Version\ 1} to \texttt{Version\ 2}:

\begin{itemize}
\item
  Re-defined the structure and content of \texttt{TC=28},
  \texttt{TC=29}, and \texttt{TC=31} messages.
\item
  Introduced two additional NIC Supplement Bit
\item
  \texttt{NICa} is defined in operational status messages
  (\texttt{TC=31})
\item
  \texttt{NICb} is defined in airborne position messages
  (\texttt{TC=9-18})
\item
  \texttt{NICc} is defined in operational status messages
  (\texttt{TC=31})
\item
  Introduced an additional ``Horizontal Containment Radius
  (\texttt{Rc})'' within \texttt{NIC=6} / \texttt{TC=13}
\end{itemize}

\subsection{Identify the ADS-B
Version}\label{identify-the-ads-b-version}

There are two steps to check the ADS-B version, this is due to the fact that ADS-B \texttt{Version\ 0} is not included in any message.

\begin{enumerate}
\def\labelenumi{\arabic{enumi}.}
\item
  Step 1: Check whether an aircraft is broadcasting ADS-B messages with   \texttt{TC=31} at all. If no message is ever reported, it is safe to assume that the version is \texttt{Version\ 0}
\item
  Step 2: If messages with \texttt{TC=31} are received, check the version numbers located in the 41-43 bit of the payload (or 73-75 bit of the message).
\end{enumerate}

After identifying the right ADS-B version for an aircraft (which does not change often), one can decode related \texttt{TC=28}, \texttt{TC=29}, and \texttt{TC=31} messages accordingly.
