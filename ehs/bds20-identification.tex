\section{Aircraft identification (BDS
2,0)}\label{aircraft-identification-bds-20}

Similar to an ADS-B aircraft identification message, the callsign of an
aircraft can be decoded in the same way. For the 56-bit MB (message,
Comm-B) field, information decodes as follows:

\begin{verbatim}
+------------+------+------+------+------+------+------+------+------+
|  BDS2,0  8 | C1,6 | C2,6 | C3,6 | C4,6 | C5,6 | C6,6 | C7,6 | C8,6 |
+------------+------+------+------+------+------+------+------+------+
  0010 0000      6 bits
\end{verbatim}

Here, 8 bits are 0010 0000 (2,0 in hexadecimal) and the rest of chars
are 6 bits each. To decode the chars, the same char map as ADS-B is
used:

\begin{verbatim}
'#ABCDEFGHIJKLMNOPQRSTUVWXYZ#####_###############0123456789######'

\end{verbatim}

Example:

\begin{verbatim}

MSG:  A000083E202CC371C31DE0AA1CCF
DATA:         202CC371C31DE0

BIN:  0010 0000 001011 001100 001101 110001 110000 110001 110111 100000
HEX:     2    0
DEC:             11     12     13     49     48     49     55     32
CHR:             K      L      M      1      0      1      7      _

ID:   KLM1017
\end{verbatim}
