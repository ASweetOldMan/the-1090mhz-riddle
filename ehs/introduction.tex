\section{Enhanced Mode-S Basics}\label{introduction}

\subsection{Downlink Format and message
structure}\label{downlink-format-and-message-structure}

DF 20 and DF 21 are used for downlink messages.

The same as ADS-B, in all Mode-S messages, the first 5 bits contain the
Downlink Format. The same identification process can be used to discover
EHS messages. So the EHS messages starting bits are:

\begin{verbatim}
DF20 - 10100
DF21 - 10101
\end{verbatim}

The message is structured as follows, where the digit represents the
number of binary digits:

\begin{verbatim}
+--------+--------+--------+--------+------------+---------+------------+
|  DF 5  |  FS 3  |  DR 5  |  UM 6  |  AC/ID 13  |  MB 56  |  AP/DP 24  |
+-----------------+--------+--------+------------+---------+------------+
|   <---------+        32 bits       +-------->  |


DF:     downlink format
FS:     flight status
DR:     downlink request
UM:     utility message
AC/ID:  altitude code (DF20) or identity (DF21)
MB:     message, Comm-B
AP/DP:  address parity or data parity
\end{verbatim}

Except for the DF, the first 32 bits do not contain useful information
to decode the message. The exact definitions can be found in ICAO annex
10 (Aeronautical Telecommunications).

\subsection{Parity and ICAO address
recovery}\label{parity-and-icao-address-recovery}

Unlike ADS-B, the ICAO address is not broadcast along with the EHS
messages. We will have to ``decode'' the ICAO address before decoding
other information, and ICAO is hidden in the message and checksum.

Mode-S uses two types of parity checksum Address Parity (AP) and Data
Parity (DP). Majority of the time Address Parity is used.

\subsubsection{Address Parity}\label{address-parity}

For AP, the message parity field is produced by XOR-ing the ICAO address with the message
data CRC checksum. So, to recover the ICAO address, simply reversing the
XOR process will work, shown as follows:

\begin{verbatim}
+-------------------------------+  +--------------------+
|   DATA FIELD (32 OR 88 BIT)   |  |    PARITY BITS     |
+--------------+----------------+  +--------------------+
               |
               |                           XOR
               v
+--------------+-----------+       +--------------------+
|          ENCODER         |  +--> | CHECKSUM (24 BITS) |
+--------------------------+       +--------------------+
                                            ||
                                   +--------------------+
                                   |   ICAO (24 BITS)   |
                                   +--------------------+
\end{verbatim}

An example:

\begin{verbatim}
Message:      A0001838CA380031440000F24177

Data:         A0001838CA380031440000
Parity:                             F24177

Encode data:  CE2CA7

ICAO:    [F24177] XOR [CE2CA7] => [3C6DD0]
\end{verbatim}

For the implementation of the CRC encoder, refer to the pyModeS library
\texttt{pyModeS.common.crc(msg,\ encode=True)}

\subsection{BDS (Binary Data Selector)}\label{bds-comm-b-data-selector}

In simple words, BDS is a number (usually a 2-digit hexadecimal) that
defines the type of message we are looking at. Both ADS-B messages and
other types of Mods-S messages are all assigned their distinctive BDS
number. However, it is \textbf{nowhere} to be found in the messages.

When SSR interrogates an aircraft, a BDS code is included in the request
message (Uplink Format - UF 4, 5, 20, or 21). This BDS code is then used
by the aircraft transponder to register the type of message to be sent.
But when the downlink message is transmitted, its BDS code is not
included in the message (because the SSR knows what kind of message it
requested). Good news for them, but challenges for us.

Here are some BDS codes that we are interested in, where additional
parameters about an aircraft can be found:

\begin{verbatim}
BDS 2,0   Aircraft identification
BDS 2,1   Aircraft and airline registration markings
BDS 4,0   Selected vertical intention
BDS 4,4   Meteorological routine air report
BDS 5,0   Track and turn report
BDS 6,0   Heading and speed report
\end{verbatim}
